\section{Questions et réponses}



\begin{enumerate}
	\item "Mécanisme empêchant l'auto fécondation ? Schéma floral !"

	\item "Comparer les gamétophytes des Sélaginelles et des Filicophytes."

	\item "Comparez le cycle de vie d'un Coleochète et de Pellia Epiphylla. A partir de cette comparaison (schémas précis), établissez une liste des apomorphies des Embryophytes."

	\item "Comparer les cycles de chlamydomonas et les diatomées."

	\item "Comparer le sporophyte d'un bryophyte, marcantiophyte et anthocérophyte."

	\item "Comparer le cycle des Licophytes et des Cycadophytes." Il m'a un peu questionné sur l'intérêt d'étudier les Cycadophytes en cours.

	\item "Mégasporophylle: tendance évolutive, replacer dans l'évolution, est-ce un caractère dérivé propre?"

	\item "Évolution des gametophytes mâles chez les embryophytes, dessins à l'appui depuis leur origine."

	\item "Comparer la structure de l'ovule des Angiospermes et des Pinophytes, établir les ressemblances et les différences et s'appuyer sur des schémas précis. Parler des différences entre les cycles et plus particulièrement les phases gamétophytiques des Filicophytes et des Sélaginelles."

	\item "Comparer la gamie chez les Lycophytes et chez les cycadophytes."

	\item "Expliquer le mécanisme de l'allopolyploidie et donner des exemples."

	\item "Marchantiophytes Antocérophytes et Bryophytes sont paraphylétiques, explicitez à l'aide de schémas précis."

	\item "Comparer les sporophytes d’Ulva lactuca et Porphyra. Prenez en compte les critères cytologique, biochimique, morphologique, écologique, et tout ce qui peut être utile." Il a appuyé sur les thylacoïdes et des phycobilisomes (dessin... --'), parler des parois des cellules, du groupe le plus archaïque...

	\item "Comparer le cycle de chlamydomonas et celui des diatomées en utilisant des schémas les plus précis possibles."
 $\Rightarrow$ Ressemblances : reproduction asexuée la plupart du tps sauf à certains moment, cycle monogénétique, anisogamie physiologique et isogamie morphologique.
 $\Rightarrow$ Différences : chl : haploide, planogamie dia : diploide, planogamie, 2 indiv forme 1 indiv. 
Ensuite, il m'a demandé de donner un autre cycle monogénétique haploîde (spirogyre), un autre cycle monogénétique diploïde (fucus). 
Ce cycle est-il fréquent dans le règne végétal? Non
 A quel clade appartiennent les diatomées? Lignée brune  $\rightarrow$ straménopile $\rightarrow$ chrysophyte $\rightarrow$ diatomophycé. Les types de gamie ( cystogamie : toute le cellule fusionne, plasmogamie : cytoplasme qui fusionne, caryogamie : noyau qui fusionne),  es les types de méiose(zygotique, gamétique),...

	\item "Expliquer l'évolution des différents caractères  cytologique des algues vertes aux embryophytes. + en quoi on t ils influencés la "formation" de l'arbre phylo. (+dessins et légendes...)."
$\Rightarrow$ Mitose fermée $\rightarrow$ ouverte $\rightarrow$ phragmoplaste $\rightarrow$ plasmodesme (oosphère reste sur gamétophyte) $\rightarrow$ parenchyme (zygote reste sur gamétophyte et est nourrit par celui-ci) $\rightarrow$ organe producteur de gamètes et phase diploïde multicellulaire.
$\Rightarrow$ Influence algue verte dans l’arbre : pas de taxon " algue " car algues vertes sont plus proches des plantes terrestres que des algues brunes

	\item "Expliquer les modalités de la gamie chez les embryophytes avec dessins et légendes." Archégone (cmt ça fonctionne), anthéridie (idem), embryon qui reste sur gamétophyte protection, nutrition, etc

	\item "Pq dit-on des hépatiques, des anthocérotes et des mousses qu'ils sont des groupes paraphylétiques?

Ils ne sont pas monophylétiques car ce taxon se baserait sur des plésiomorphies (carac ancestraux) : plante de petite taille, gamétophyte dominant, sporophyte parasite du gameto, app végétatif = thalle, pas de racines ms des rhyzoides. Il veut des cycles, l'arbre de l'évolution des embryophytes avec ce qui diverge un peu partout. Il est très pointilleux sur tout ce qui concerne apomorphie, plésiomorphie, para/mono/polyphylétique en général!

	\item "Cycle du pin", d'autre ont eu le "cycle du fucus" ou une "comparaison entre la graine des gymnospermes et celle des angiospermes..." gymnosperme : accumulation des réserves avant la fécondation, tps de latence entre pollinisation et fécondation et chez cycas et ginko embryon se développe tout de suite.
 Et aussi qd vs étudiez ou même préparez la question pour l'oral faites un max de schémas avec des légendes....
 
 
	\item Parmi les propositions suivantes, lesquelles sont incorrectes ? Donnez un énoncé correct avec schéma, légendes... 

a) "le Saule et le Bouleau sont fortement apparentés car ils présentent tous deux des fleurs anémogames en chatons."
$\Rightarrow$ FAUX: saule : Salicaceae bouleau : Betulaceae On ne peut se baser sur anémogamie car c’est un caractère qui est apparu à de nombreuses reprises.
 b) "les Marchantiophytes et les Bryophytes forment un clade car ces deux groupes ont un cycle digénétique à gamétophyte dominant et ne possèdent pas de vraies racines."
$\Rightarrow$ FAUX: Pas un clade car basé sur des plésiomorphies

	\item "Algues unicellulaires, discuter des diversités avec rapport phylogénétiques plus schéma." (algues vertes : chlamydomonas ; algues brunes : diatomé, coccolithacés, Péridinien)

	\item "Comparez le système des 5 règnes de Margulis et la phylogénie moderne. Relevez et discutez les concordances et discordances entre les 2 systèmes."
 Monères : rassemble archées et bactéries, algues : algues vertes plus proches des plantes terrestres que des algues brunes, Fungi : il faut montrer parenté entre eux et animaux, Protoctistes : dispercés dans l’arbre
Je mets la réponse, ça peut toujours aider : petite remarque parce que je me suis planté là-dessus: Margulis définit ses 5 règnes mais sans les situer les uns par rapport aux autres ce qu'il veut, c'est qu'on envisage chacun des 5 règnes et qu'on détermine si ce sont des clades monophylétiques, définis par une apomorphie.
- Monères : organismes dépourvus de noyau. NON car c'est une plésiomorphie. 
- Protoctistes : Eucaryotes uni-c ou pluri-c à faible niveau de différenciation cellulaire. NON définit un groupe polyphylétique (entre autres algues brunes et algues vertes mais pas embryophytes)
- Champignons : hétérotrophes par assimilation. OUI ça correspond aux Eu-mycètes qui constituent un clade monophylétique (apomorphie des Eu-mycètes : structure en hyphes. savoir ce qu'est un hyphe)
- Végétaux : Eucar photoautotrophes pluri-c à haut degré de différence. OUI, ça correspond aux embryophytes qui sont monophylétiques (apomorphie : gamétanges. savoir dessiner)
- Animaux : Eucar hétérotrophes par ingestion pluri-c à haut degré de diff. OUI, ça correspond aux Métazoaires qui constituent un groupe monophylétique. (apomorphies: celles de Margulis sont bonnes)

(page 25 sylla des algues) savoir placer sur l'arbre de l'évolution ce à quoi correspondent les 5 "règnes" : eucaryotes, procaryotes, algues brunes / rouges / embryophytes, mycètes/eumycètes, choano-organismes, choanoflagellés et métazoaires.

	\item "Les Sellaginelles, plan de construction et mode de vie + cycle." (Ne pas confondre avec les Filicophytes !).

	\item "Replacez dans une évolution phylogénétique les algues vertes." J'ai donc crée le petit arbre phylo et ai expliqué les diverses évolutions :(streptophytes, phragmoplastophytes,plasmodesmophyte, parenchymophyte,....) et il a un peu fouillé sur les charophyte (pourquoi on en retrouve à l'état fossile car présence de cellulose?? Quel est la convergence évolutive avec les embryophytes? Ca c'est le bec à 5 dents qui préfigure le col de l'archégone.)

	\item "Replacez sur une ligne de temps les évènements suivant :
\begin{enumerate}
	\item Chloroplastes (2) $1,5.10^9$
	\item Double fécondations (6) $360.10^6$
	\item Lignigne (3)
	\item Archegone (4) $460.10^6$
	\item Photosynthèse oxygénique (1) $3 .10^9$
	\item Ovaire (5) $360.10^6$"
\end{enumerate}

Donner les dates (!) et expliquez un peu."

	\item "Comparer le gamétophyte des lyco et filicophytes."
Lycophyte : prothalle male 1 anthéride avec 8 cellules stériles et 4 cellules spermatogènes; prothalle femelle : cénocytique avec célularisation à partir du haut endosporie et hétéroprothallisme
Filicophyte : prothalle en cœur et les gamétanges sur le même. Héterothalisme mais exosporie avec le col du thalle femelle qui dépasse et une partie du mâle aussi.

	\item "Algues rouges, tout expliquer: j’ai tout expliquée avec le cycle de porphyra.

	\item "Mode de vie et cycle de développement des hépatiques."

	\item "Placer ces événements dans l'odre chronologique : apparition phragmoplaste(2), fixation de l'azote(1), apparition de la meiose(3), apparition de la siphonogamie, apparition du pollen..."


	\item "Parler moi des stramenophiles, de leur morphologie, de leur écologie et de leur importance pour l'homme" (diatoméé, Fucus:alginates, sel minéraux, engrais) et il est un peu parti en couille sur les alginates (polymère de monosaccharide avec des ions alcalin, de là, il te demande des trucs sur ton ion alcalin et c'est quoi comme liaison chimique?....euuuuuuuuuuuuuh (enfait, c'était ionique tout simplement)

	\item a) "Le bouleau et le saule sont fortement apparenté puisqu'ils ont tous les 2 des chatons." $\Rightarrow$  Faux: expliquer ce qu'est un chaton (inflorescence, grain de pollen, expliquer le gynécé), pourquoi ils ont évolué comme ça (anémogamie), donner d'autre exemples de familles qui sont anémogames (j'ai dit poaceae mais il attendait gymno)
b) "les marchantiophytes et les bryophytes forment-ils un clade du fait qu'ils possèdent tous deux un gamétophyte dominant et des rhizomes?" Faux: déjà c'est un groupe para mais de plus on ne peut pas utiliser ces caractères car ce sont des plésiomorphies et non des apomorphies. Quel est l'embryophyte le plus ancestral?...

	\item "Comparaison cycle filicophytes et gymnospermes."

	\item "comparez le sporophyte de Ulva lactuca et de Porphyra" 
Je me suis cassée la tête à retrouver les deux cycles en entier, et pis quand je suis passée, il m’a dit que ce n'était pas nécessaire, et qu'on parlait seulement du sporophyte.

	\item "Décrivez (inventez) le cycle d'une algue unicellulaire qui SERAIT  isomorphe,
digénétique, haplodiploide" et je ne sais plus la dernière caractéristique...

	\item "Expliquer les innovations qui ont permis de passer d'un mode de vie aquatique à un mode de vie terrestre. Illustrer par des schémas."

	\item "Faites un exposé phylogenétique sur la diversité des cellules flagellées dans le monde végétal + dessins"

	\item "Le clade des protoctistes de margulis est-il un vrai clade ?Replacez les différents groupes des protoctistes dans la phylogénie actuelle." NON car euca unicel et multicel à faible niveau de différenciation cell (champi inf, algues, protozoaire)

	\item "Comparez la gamie entre les cycadophytes et les coniférophytes" : il y siphonogamie chez les 2, mais chez cyca, les gametes sont encore nageuses et relachées dans un liquide dans l'ovule...

	\item "Fixation biologique de l'azote : définir, préciser les organismes qui le font et leurs relations phylogénétiques." Il faut bien sûr parler des cyanobactéries (c'était un avantage pour elles de transformer N2 en NH4+ pour construire leur protéines ; cette aptitude est apparue au moment où les sources d'azote -a.a. dans l'océan primitif- ont diminués à cause des organismes hétérotrophes en grand nombre). (Lichens) Et il faut expliquer la symbiose Rhizobium-Fabaceae (il veut qu'on dise que c'est apparu une seule fois dans l'évolution et que cette réaction est difficile à réaliser parce que l'enzyme nitrogénase a besoin de Mo2+ comme cofacteur et que celui-ci est rare dans la nature). Il m'a fait parler de chimie puis de procaryotes et des fagales/fabales/rosales et de l'anatomie d'une nodosité etc.

	\item "Relevez les différentes gamies qui ne font pas intervenir de cellules flagellées et expliquez la fusion des gamètes"  porphyra, diatomé, spyrogyre.

	\item "Comparer les carpospores des basidimycètes et des ascomycètes."

	\item "Sporange définition, illustrer les différents sporanges chez les embryophytes avec caractères archaïques et dérivés."

	\item "Expliquez l'évolution des gynécées chez les rosales. Expliquez à l'aide de schémas."
	
	\item "Expliquer les innovations qui ont permis de passer d'un mode de vie aquatique à un mode de vie terrestre. Illustrer par des schéma." Il fallait juste expliquer l'évolution des plantes pour arriver aux plantes terrestres (sortie de l’eau donc couche protectrice pour éviter le dessèchement, apparition  de vaisseaux, différentiation cellulaire avec la communication, méristèmes... et ce genre de truc)

	\item "Classez ces organismes et montrer les relations phylogénétique +dessiner le structures:
-Cladonia
-rouille du blé
-amanite phalloïde"


	\item "Discutez, parler moi de Cooksonia Marchantia et de la Sphaigne, relier les trois organismes dans la phylogénie des embryophites avec les différents caractères et groupes." Puis j’ai dû parler des 3 organismes en général. Pour Cooksonia je ne savais pas précisément ce que c'étais mais lui ai expliqué que c'était un organisme désormais à l'état fossilisé qui était assez proche des lycophites et inclus dans les polysponrangiophytes.

	\item "Les différents cycles de reproductions chez les végétaux phototrophe, la plus archaïquL, et le développement et avantages des autres."

	\item " Montrez la diversité des gynécées au sein des angiospermes, à l'aide d'un arbre phylogénétique et présentez les caractères dérivés et archaïques."

	\item "Expliquer à travers des schémas et un arbre phylogénétique l'évolution des grains de pollen chez les Spermatophytes ;-p !"  Il voulait qu'on lui explique la structure du grain de pollen des gyncophytes, pinophytes, angiospermes, ... et les cycles associés !

	\item "Faites un exposé phylogenétique sur la diversité des cellules flagellées dans le monde végétal + dessins"

	\item "Retracez les cycles de reproduction de la lignée verte"

	\item "Le clade des protoctistes de margulis est-il un vrai clade ?, replacez les différents groupes des protoctistes dans la phylogénie actuelle"
Alors la réponse est non, car le clade des protoctiste regroupe les "eucaryotes unicellulaires et pluricellulaires à faible niveau de différenciation", il est basé sur des plésiomorphies (unicellulaires, faible niv de différenciation), et il regroupe des groupes incomplets (algues vertes sans les plantes terrestres). Pour la suite de la question, il fallait redessiner l'arbre phyllo des eucaryotes (!!!) et explique le développement.

	\item "Comparez la gamie entre les cycadophytes et les coniférophytes" 
Il y siphonogamie chez les 2, mais chez cyca, les gametes sont encore nageuses et
relachées dans un liquide dans l'ovule...

	\item "Relevez les différentes gamies qui ne font pas intervenir de cellules flagellées et expliquez la fusion des gamètes"

	\item "Comparer les carpospores des basidiomycètes et des ascomycètes."

	\item "Sporange définition, illustrer les différents sporanges chez les embryophytes avec caractères archaïques et dérivés."

	\item "Comparer les plastes des algues rouges et des embryophytes. dessin bien sûr!!" En gros faut connaitre l’endymbiose mais ça suffit pas, faut connaitre le contenu d’une cyano, par exemple les thyllakoides des algues rouges ne sont pas disposés en grana;quand les phicobilisomes des algues vertes ont disparus?"

	\item "Parler des gamies où aucun gamète flagellés intervient, comment ceux-ci font pour se rencontrer? Et nommer ces gamies." En gros c'est tout simple mais il faut connaître tous les cycles de reproduction et parler des :
	\begin{enumerate}
		\item myxomycètes avec la phase où les gamète sont des amibes
		\item champi asco/basidio
		\item algues rouge
		\item algues verte: cas de la spyrogyre
		\item algues brune: cas des diatomés
		\item conipherophyte: bah formation d’un tube pollinique
	\end{enumerate}

	\item "1 cycle de mycétozoaire" Je lui ai retapé tout le cycle avec tout les détails et après il m'a juste demandé pourquoi les mycétozoaires ressemblent aux animaux et aux champi.

	\item "Évolution des gametophytes males des embryophytes.... tout schéma, caract. dérivé propre, évolution, rôle,..."

	\item "Comparez le sporophyte des marchantio, anthocéro et bryophytes, quels sont leurs caractères ancestraux et dérivés, appuyez-vous sur des dessins."
ca ne semble pas trop dur comme question mais alors, il demande une précision dans le dessin! Le sporophyte qu’on a dessiné en cours dans le cycle du groupe est pas du tout suffisant, il veut un vrai dessin avec les différentes cellules qui le composent etc. Il vaut donc mieux jeter un coup d'œil dans le livre de référence. Puis évidement, pour ce genre de questions, prévoyez qu'il vous demande tout le cycle, avec dans le cas de cette question beaucoup de détails avant et après le sporophyte. Perso j'avais une petite faute pour le cycle de marchantia polymorpha dans mon cours (le zygote ne subit pas de méiose, comme j'avais dessiné, il se détache de l'archégoniophore et germe: c'est logique et y a peut-être que moi qui ai mal vu et mal compris le cycle).

	\item "Comparer les charophytes et les coloechatophytes du point de vue phylogénétique"
	
	\item "Est-ce que ces taxons sont légitimes : fougères, spermatophytes et dicotylédones, argumentez et discuter ."
	
	\item "Parler de l’évolution du megasporophylle chez les embryophytes (schéma, tout ça..)"
	
	\item "Décrire la variété des grains de pollen (attention aux dessins) et leur position dans l'évolution. Grains de pollen monoaperturés chez les monocots puis triaperturés chez les dicots." Je m'en suis sorti en définissant le pollen et son origine, puis en parlant des différentes oogamies et siphonogamies (tube pollinique etc., chez les gymnos surtout). En fait, il voulait surtout que je dessine des grains de mollen mono- et triaperturés (je ne savais pas trop à quoi ça ressemble en fait, et il faut faire gaffe à dessiner le tégument, les cellules ciliées ou non...). Enfin, on a parlé de la différence entre pollen dans l'entomogamie (collant) et l'anémogamie (pas collant et plus petit pour des raisons de probabilités de la pollinisation). Au final, c'est passé mais il a râlé sur mes dessins pas assez précis. cacaaaaa
	
	\item "Décrire les plastes des glaucophytes, algues rouges et algues brunes d'un point de vue pigmentaire et morphologique, la théorie à l'origine de ces plastes."
	
	\item "L'apparition et l'évolution de l'ovule et du carpelle"
	
	\item "Comparer les sporophytes de ulva lactuca et porphyra du point de vue cytologique, biochimique, écologique et estructural, aidez vous de schemas précis"
	
	\item "Comparez les cycles d'un coleochaete et de pellia epiphylla. Donnez les apomorphies des embryophytes. S'appuyez sur des schéma précis"
	
	\item "Définissez précisément les termes ovule et carpelle + expliquez comment ils se sont formés au cours de l'évolution à l'aide de schémas précis et légendés."
	
	\item "Heteroprothallie chez les embryophytes. Est-ce arrivé un seule fois chez les embryophytes. Discutez à l'aide de schéma et arbre phylogénétique."
	
	\item "Haplodiploïdisation: expliquer le mécanisme plus donner un exemple concret. L espèce qui en est déduite est-elle une nouvelle espèce? (Et peut elle se reproduire avec ses parents mais j suis plus sûre si c était sur la fiche ou non)"
	
	\item "Tout d'abord, tu définis le grain de pollen (gamétophyte mâle). Tu expliques dans quel groupe et avec quoi il apparaît (Spermatophyte, apparâit avec l'ovule). Tu peux aussi dire qu'il est une "amélioration" des microspores (cf Lycophytes, filico). Ensuite, aiguille Meerts vers les 2 grands types dans les Angios (mono-aperturé et tri-aperturté)."
\end{enumerate}